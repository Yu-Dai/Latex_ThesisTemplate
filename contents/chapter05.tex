% !TeX root = ../main.tex
\titlespacing{\chapter}{0cm}{-2cm}{0cm}
\chapter{結論}
本研究提出對於未知效能之光譜晶片模組,開發一套快速的通用演算法與自動化校正程式,對於不同的波形與強度落差,皆能有效計算出晶片之波長校正方程式且不失準確度,並且將每個晶片的優劣程度量化,使後續晶片用途之挑選能有定量依據與快速的效率。
\par
為縮短光譜演算時之計算量並提高整體校正速度,使用ROI SCAN方式找出光譜的ROI,並以Auto Scaling演算法調整光譜強度,以解決晶片效率不同的問題。為了在不同光源不同波形不同效率下皆能精確的找出波峰,本研究將波形先以希爾伯特轉換找出粗略波峰位置,並透過區域最大值與最小值有效的對光譜進行分區,並對每一分區以勞倫茲函數模型擬合出精確的波峰位置,並以波峰間距離不變理論過濾無效波峰,最後擬合出精準的波長校正方程式,並在最後以白光與汞燈光譜進行光譜晶片優劣的量化,提供了後續的晶片挑選依據。
\par
透過第四章的實驗結果可以看出,經過此自動化校正程式所計算出的波長校正方程式,轉換後精確度RMSE約落在3nm內,且透過波長校正方程式轉換至波長空間中的白光光譜十分精準,因此可以看出本研究的演算法同時具有快速、通用與不失精確度的重要優勢。
\par
本研究結果可以有效提高光譜晶片生產的產能,並減少人力的耗損,將過往人力計算時單一晶片耗時半小時至四十分鐘大幅縮短至單一晶片校正僅需五分鐘,並透過提供的晶片優劣量化數據,提供晶片用途的依據並可以反饋給晶片製程者參考。
\par
由第四章的不同方法校正方程式精確度比較中可以看出,結合校正是所有校正中最為精確的方法,但由於在更高波段區域仍無參考波峰,因此晶片若想應用於1000nm以上超高波段時,將會有越來越大的偏差,未來若能再加入更高波段之光源,能使波長校正方程式之精確度與波段通用性都大幅提升。