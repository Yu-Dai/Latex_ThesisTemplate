% !TeX root = ../main.tex
\titlespacing{\chapter}{0cm}{-2cm}{0cm}
\chapter{緒論}
%\chapter{中國文學}
\section{前言}
光譜分析可以得到許多表面無法看出的深度訊息,因物體所發的光可由多樣因素組成,也就是物體發光的原因可能同時由眾多因素造成,這時光譜組成十分豐富,比如紅光與綠光組合出的黃光與單一波長黃光所形成的光譜大為不同,前者可以找到兩個波峰,而後者只有一個波峰。所以光譜因其豐富的訊息,使得光譜檢測一直熱門且廣泛的應用於各個領域,如食品檢測、毒品檢測、水質評斷、生醫應用。\par
由於光譜檢測具有非破壞性、非接觸性、高靈敏度、檢測快速的幾個主要優點,又因為物質光譜如同物質的指紋一般,每種物質皆有不重複的獨特光譜資訊,所以光譜能讀出的非常豐富且精準的資訊,就連當今最棘手的新冠肺炎病毒快篩也使用光譜檢測判定。
\section{文獻探討}

光譜分析用途廣泛,但在光譜晶片製成後獲得光譜前,必須經過一道稱為波長校正的流程,將像素空間的訊息精準的轉換至波長空間。波長校正需要波峰於像素空間與波長空間的標準位置,由於波長校正常以光源的已知特定波長的波峰作為校正參考點,因此波長校正的重點可以說是在於如何找出這些特定波峰精準的像素位置,所以找波峰便成為波長校正的重點之一,R. Mabrouki和B. Khaddoumi (2014)\cite{Hilbert-1}與S. K. Mukhopadhyay和M. Mitra (2013)\cite{Hilbert-2}皆使用希爾伯特轉換應用於醫療心電圖之信號分析,對於信號的峰值特徵提取十分有效,信號分析中時常使用希爾伯特轉換技術偵測波峰與分割訊號,這一技術對於波長校正而言雖然無法找出精確波峰位置,但由於能找出波峰粗略位置,對於分割訊號提供擬合十分有效。波長校正必須透過模型擬合才能找出精確的波峰像素位置,通常分割後的數據最常以高斯模型擬合逼近其原始波形,但是否存在更好的模型也是本研究的目標之一。而C. Bi與L. Li (2008)\cite{Lorentz-00}使用勞倫茲模型擬合原子與晶粒光譜,C. Li與Y. Li (2009)\cite{Lorentz-01}也使用勞倫茲模型擬合光譜,並以模型參數計算出精確的波峰與半高全寬,且研究結果顯示果十分優良,因此本研究在汞氬、LASER與藍光晶粒之光譜使用勞倫茲模型擬合,並依勞倫茲函數找出精確的波峰位置。
\par
因光譜晶片有時會使光譜波形在某些波段基線上揚,而此基線並無規則可循且造成波形受到干擾而影響分析。W. Shan與P. Liu (2019)\cite{BaselineCNN}提出對於不規則波形之基線處理方式,以CNN神經網路消除基線並除去光譜波形之雜訊,此方法雖然效果優異但過程較為複雜且耗時較久,L. P. Chang (2018)\cite{LP-BaseLine}使用小波轉換找出波形基線特徵點,並擬合出基線曲線並扣除之,此方法相較於CNN神經網路方式計算較為容易且快速,因此對於本研究提升校正速度的目的較為符合。

\section{研究動機}
當光譜晶片被製作出來後,會交由技術人員人工波長校正,經過找尋中心ROI、亮度調整、波峰查找與計算、多項式擬合等繁複的流程才能得出校正結果,然而在未來光譜分析將被大量且廣泛運用的時代中,這無疑是非常沒有效率的。\par
本研究希望將影像視為一巨大矩陣,並以一個與ROI相同大小矩陣快速掃描並找出中心ROI,以降低人眼判定亮度中心的誤差。而在晶片效能仍未知的情況下,雖以同一光源作為輸入,但因晶片差異仍時常發生強度過曝或過低的情形。強度的不確定造成波峰尋找的困難,過曝時造成有效擬合點減少,使得擬合出的波峰不精確,而過低時則難以分辨為波峰或為雜訊,因此本研究希望以動態並自動調整影像感測器參數,以得到強度固定之數據。\par
雖以調整亮度參數使數據最大強度幾乎固定於同一數值,得以有效解決過曝問題,但對於光譜過低的數據仍難以解決,因此本研究希望在此階段能找出一種演算法,雖精準度不及已知特光譜下人工辨識找出所有波峰,但對於所有光譜晶片皆通用的演算法。因此本研究旨在克服以上阻礙並大幅度降低校正的耗時、去除人為誤差,且在不失精確度情況下具有通用性。

\section{論文架構}
以下為本文架構,以條列式來簡短描述各章之目的。\par
第一章:緒論。文獻探討以及介紹本篇論文的研究動機與目的。\par
第二章:方法與理論。介紹本研究使用之不同光源與其適合之擬合模型,並介紹本研究使用之理論。\par
第三章:自動波長校正流程。軟體架構流程總覽與本研究之演算法開發時所遭遇問題與解決方法詳細分析與介紹。\par
第四章:實驗結果。實際將晶片以本研究開發之自動化程式校正,並比較不同方法結果之準確度。\par
第五章:結論。總結本篇論文及未來改善方向。